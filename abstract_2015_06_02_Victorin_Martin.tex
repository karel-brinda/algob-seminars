\documentclass[a4paper,12pt]{report}

\usepackage{algobseminar}

\usepackage{fontspec}
\usepackage{a4wide}


\begin{document}

\algob

\title{Detection of abnormal transcripts expressed in cancer cells}

\speaker{Victorin Martin}{Institut Curie}

\date{Tuesday, June 2, 2015 at 11:00 in 4B084}
\place{Laboratoire d'Informatique Gaspard-Monge\\Université Paris-Est Marne-la-Vallée\\5 bd Descartes, Champs-sur-Marne}

\abstract{
Next Generation Sequencing (NGS) technologies allow to ``cheaply'' generate large amounts of data making it possible to discover anomalies that were previously out of reach. In particular, paired-end transcriptome sequencing allows us to identify gene fusions, caused by abnormal chromosomal rearrangements, which are known to play an important role in cancers.  However, due to the level of noise in RNA-seq data, existing fusion detection tools tend to generate a lot of false positive making their use difficult for biologists. We propose to use probabilistic graphical models to combine RNA-seq data \& prior information for this detection problem. It aims at reducing the number of false positive while allowing us to add biological {\em a priori} in the model.

This work is done with Valentina Boeva.
}

\end{document}
