\documentclass[a4paper,12pt]{report}

\usepackage{fontspec}
\usepackage{hyperref}
\usepackage{a4wide}

\newcommand{\algob}{
	\begin{center}
	\large
	AlgoB - the seminar you can't miss\\
	{\url{http://algob.univ-mlv.fr}}
	\end{center}
	\medskip
}

\renewcommand{\title}[1]{
	\begin{center}
	{\textbf{\huge #1}}
	\end{center}
}

\newcommand{\speaker}[2]{
	\begin{center}
	{\textbf{\Large #1}}\\
	{#2}
	\end{center}
}

\renewcommand{\date}[1]{
	\begin{center}
	{\large #1}
	\end{center}
}

\newcommand{\place}[1]{
	\begin{center}
	{#1}
	\end{center}
}


\renewcommand{\abstract}[1]{
	\bigskip
	{
	\textbf{Abstract:}
	#1
	}
}

\thispagestyle{empty}



\begin{document}

\algob

\title{Finding a tree in a phylogenetic network}

\speaker{Philippe Gambette}{LIGM Paris-Est}

\date{Tuesday, May 26 at 11:00 in 4B084}
\place{Laboratoire d'Informatique Gaspard-Monge\\Université Paris-Est Marne-la-Vallée\\5 bd Descartes, Champs-sur-Marne}

\abstract{
A fundamental problem in the study of phylogenetic networks is to determine whether or not a given phylogenetic network contains a given phylogenetic tree. We develop a quadratic-time algorithm for this problem for binary nearly-stable phylogenetic networks. We also show that the number of reticulations in a reticulation visible or nearly stable phylogenetic network is bounded from above by a function linear in the number of taxa. Finally, we also give results on this tree containment problem for other classes of phylogenetic networks.

This work was done with Andreas Gunawan, Anthony Labarre, Stéphane Vialette and Louxin Zhang.
}

\end{document}