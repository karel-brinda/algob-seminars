\documentclass[a4paper,12pt]{report}

\usepackage{algobseminar}

\usepackage{fontspec}
\usepackage[margin=1in]{geometry}


\begin{document}


\algob

\title{Sorting with forbidden intermediates}

\speaker{Anthony Labarre}{LIGM Paris-Est}

\date{Tuesday, June 16, 2015 at 11:00 in 4B084}
\place{Laboratoire d'Informatique Gaspard-Monge\\Université Paris-Est Marne-la-Vallée\\5 bd Descartes, Champs-sur-Marne}

\abstract{
A wide range of applications, most notably in comparative genomics, involve the computation of a shortest sorting sequence of operations for a given permutation, where the set of allowed operations is fixed beforehand. Such sequences are useful for instance when reconstructing potential scenarios of evolution between species, or when trying to assess their similarity. We revisit those problems by adding a new constraint on the sequences to be computed: they must avoid a given set of forbidden intermediates, which correspond to species that cannot exist because the mutations that would be involved in their creation are lethal. We focus on the case where the only mutations that can occur are exchanges of any two elements in the permutations, and give a polynomial time algorithm for solving that problem when the permutation to sort is an involution.

Joint work with Carlo Comin, Romeo Rizzi and Stéphane Vialette.

}



\end{document}
